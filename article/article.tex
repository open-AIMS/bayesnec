\documentclass[
]{jss}

\usepackage[utf8]{inputenc}

\providecommand{\tightlist}{%
  \setlength{\itemsep}{0pt}\setlength{\parskip}{0pt}}

\author{
Rebecca Fisher\\Australian Institute of Marine Science \And Diego R.
Barneche\\Australian Institute of Marine Science \AND Gerard
Ricardo\\The University of Queensland \And David Fox\\Environmetrics
Australia
}
\title{\pkg{bayesnec}: An R Package for C-R Modelling and Estimation of
No-Effect-Concentrations}

\Plainauthor{Rebecca Fisher, Diego R. Barneche, Gerard Ricardo, David
Fox}
\Plaintitle{bayesnec: An R Package for C-R Modelling and Estimation of
No-Effect-Concentrations}
\Shorttitle{C-R Modelling With \pkg{bayesnec}}

\Abstract{
The \pkg{bayesnec} package in R has been developed to fit
concentration(dose) - response curves (C-R) to toxicity data for the
purpose of deriving no-effect-concentration (\emph{NEC}),
no-significant-effect-concentration (\emph{NSEC}), and
effect-concentration (of specified percentage `x', \emph{ECx})
thresholds from non-linear models fitted using Bayesian MCMC fitting
methods via \pkg{brms} \citep{Burkner2017, Burkner2018} and \pkg{rstan}
\citep{rstan2020}. In \pkg{bayesnec} it is possible to fit a single
model, custom model-set, specific model-set or all of the available
models. When multiple models are specified the \texttt{bnec} function
returns a model weighted average estimate of predicted posterior values.
A range of support functions and methods are also included to work with
the returned single, or multi- model objects that allow extraction of
raw, or model averaged predicted, \emph{NEC}, \emph{NSEC} and \emph{ECx}
values and to interrogate the fitted model or model-set. The statistical
methods used mean that the uncertainty in derived threshold values can
be robustly quantified, including uncertainty in individual model fits
to the underlying observed data, as well as uncertainty in the model
functional form.
}

\Keywords{ecotoxicology, effect concentration, threshold
derivation, \textit{NEC}, \proglang{brms}}
\Plainkeywords{ecotoxicology, effect concentration, threshold
derivation, NEC, brms}

%% publication information
%% \Volume{50}
%% \Issue{9}
%% \Month{June}
%% \Year{2012}
%% \Submitdate{}
%% \Acceptdate{2012-06-04}

\Address{
    Rebecca Fisher\\
    The Indian Ocean Marine Research Centre,\\
University of Western Australia,\\
Crawley, WA, Australia\\
    Australian Institute of Marine Science,\\
Crawley, WA, Australia\\
  E-mail: \email{r.fisher@aims.gov.au}\\
  
      Diego R. Barneche\\
    The Indian Ocean Marine Research Centre,\\
University of Western Australia,\\
Crawley, WA, Australia\\
    Australian Institute of Marine Science,\\
Crawley, WA, Australia\\
  
  
      Gerard Ricardo\\
    Australian Institute of Marine Science,\\
Townsville, Qld, Australia\\
    The University of Queensland,\\
St Lucia, Qld, Australia\\
  
  
      David Fox\\
    University of Melbourne,\\
Parkville, Victoria, Australia\\
    Environmetrics Australia,\\
Beaumaris, Victoria, Australia\\
  
  
  }

% Pandoc citation processing

% Pandoc header

\usepackage{amsmath}

\begin{document}

\hypertarget{introduction}{%
\section{Introduction}\label{introduction}}

Concentration-response (C-R) modelling (also known as dose-response
modelling or dose-response analysis) is a key tool for assessing
toxicity and deriving the toxicity thresholds used in the risk
assessments that underpin protection of human health and the
environment. It is widely used in the disciplines of pharmacology,
toxicology and ecotoxicology, where parametric non-linear regression
methods are used to model response data of interest, with the resulting
fitted models used to estimate critical thresholds of concern. These
thresholds may be used directly to assess risk \citep[e.g
see][]{fisher2018c}, or are subsequently incorporated into a broader
population-level risk assessment framework \citep[e.g.][]{Warne2015}.
Typical thresholds derived from C-R modelling include the
effect-concentration of defined percentage `x' (ECx) and the
no-effect-concentration (NEC), the latter being the generally preferred
option \citep{Fox2008, Warne2015}.

Typically C-R relationships are non-linear, which increases the
complexity of model fitting. There are a wide array of packages for
fitting non-linear models in \proglang{R}, including the \texttt{nls}
function in \pkg{stats} which uses least squares minimisation and
extensions in the \pkg{nlme} package \citep{pinheiro2021} based on
maximum likelihood estimation that also allows for the consideration
hierarchical effects. While C-R modelling can be carried out using
multi-purpose software packages, this can be cumbersome in practice,
requiring extensive manual programming to obtain the necessary, often
relatively standard outputs. The \pkg{drc} package \citep{Ritz2016} was
developed as a user friendly frequentist based solution to C-R modelling
in \proglang{R}, and is currently widely used across a range of
disciplines. \pkg{drc} implements a broad range of non-linear C-R
models, provides facilities for ranking model fit based on AIC
\citep{Burnham2002}, joint modelling of multiple response curves, and
supports a range of estimation procedures \citep{Ritz2016}.

Estimates of uncertainty in parameters and derived thresholds are
critical for effective integration of threshold estimates into risk
assessment and formal decision frameworks \citep{fisher2018c}. Bayesian
methods allow robust quantification of uncertainty with intuitive and
direct probabilistic meaning \citep{Ellison1996}, and are therefore an
excellent platform for C-R modelling in most settings. Furthermore, the
posterior samples generated through Bayesian model fitting methods
provides a rich resource that can be used to explore synergistic and
antagonistic impacts \citep{Fisher2019d, flores2021}, propagate endpoint
uncertainty \citep{Charles2020a, Gottschalk2013}, and for testing
hypotheses of no-effect \citep{Thomas2006}.

There are a wide array of packages available for Bayesian model fitting
via Monte-Carlo based simulation methods (e.g.~MCMC, HMC), including
\proglang{WinBUGS} \citep{Lunn2000}, \proglang{JAGS} \citep{Plummer2003}
and \proglang{Stan} \citep{Carpenter2017}, that seamlessly interface
with \proglang{R} through packages such as \pkg{R2WinBUGS}, \pkg{R2jags}
\citep{Su2015} and \pkg{rstan} \citep{rstan2020}. These packages
generally require many lines of code to fit even a single model, along
with extensive debugging and optimisation that is often time consuming
and requires specialist expertise. Several extension packages are
becoming available, such as \pkg{brms} \citep{Burkner2017}, that allow a
broad range of models to be easily fit using Bayesian methods through
simple \pkg{lme4}-like formula syntax. However, even with packages like
\pkg{brms}, Bayesian model fitting can be difficult to automate across
all potential usage cases, particularly with respect to specifying valid
initial parameter values and appropriate priors. In addition, as was the
motivation for the development of the \pkg{drc} package in the
frequentist setting, the \proglang{R} code required for fitting
non-linear models and extracting the relevant outputs (e.g.~NEC, ECx)
can be cumbersome in practice, and even more so in the Bayesian setting
where model fits contain many posterior samples.

The greater complexity associated with Bayesian model fitting has likely
hindered the uptake of Bayesian statistics for C-R threshold derivation
across the broader ecotoxicology and toxicology communities, who may not
have access to specialist statistical expertise \citep{Fisher2019}. The
\pkg{bayesnec} package provides an accessible interface specifically for
fitting \emph{NEC} and other C-R models using Bayesian methods. A
variety of models can be specified based on the known distribution of
the ``concentration'' or ``dose'' variable (the predictor, x) as well as
the ``response'' (y) variable. The model formula, including priors and
initial values required to call \pkg{brms} are automatically generated
based on information contained in the supplied data. A range of tools
are supplied to aid the user in interrogating model fits, plotting and
generating predicted values, as well as extracting the standard outputs,
such as \emph{NEC} and \emph{ECx} - either as a full posterior sample or
in summary form.

\hypertarget{technical-details-and-usage}{%
\section{Technical details and
Usage}\label{technical-details-and-usage}}

This project started with an implementation of the \emph{NEC} model
based on that described in \citep{Fox2010, Pires2002} using \pkg{R2jags}
\citep{Su2015}. Code for fitting this initial model was developed into
the R package \pkg{jagsNEC} and expanded to include several C-R models
and further generalised to allow a large range of response variables to
be modelled using their appropriate statistical distribution. While the
original \pkg{jagsNEC} implementation supported Gaussian, Poisson,
Binomial, Gamma, Negative-binomial and Beta response data,
\pkg{bayesnec} supports all of these in addition to the Beta-binomial
family, implemented as a custom family in \pkg{brms}. Furthermore, the
new structure implemented using \pkg{brms} means \pkg{bayesnec} can be
readily extended to include any of the \pkg{brms} families. In addition
to greater flexibility in the available response families,
\pkg{bayesnec} includes a larger range of alternative \emph{NEC} model
types, as well as most typically used smooth C-R models (such as
4-parameter logistic and Weibull models) that have no \emph{NEC} `step'
function but simply model the response as a smooth function of
concentration. We have now incorporated most of the commonly used models
in frequentist packages such as \pkg{drc} \citep{Ritz2016} (please see
the
\href{https://open-aims.github.io/bayesnec/articles/example2b.html}{Model
details} vignette for more information on the full list of models
currently available in \pkg{bayesnec}).

\hypertarget{model-specification}{%
\subsection{Model specification}\label{model-specification}}

The main working function in \pkg{bayesnec} is \texttt{bnec}. We have
attempted to make the \texttt{bnec} function as easy to use as possible,
targeting the novice R user. While the package is available on CRAN,
here we use the development version by installing directly from GitHub
using the package \pkg{devtools}:

\begin{CodeChunk}
\begin{CodeInput}
R> devtools::install_github("https://github.com/open-AIMS/bayesnec")
\end{CodeInput}
\end{CodeChunk}

\newpage

A \pkg{bayesnec} model can be fit as simply as:

\begin{CodeChunk}
\begin{CodeInput}
R> library(bayesnec)
R> data(nec_data)
R> exmp_fit <- bnec(data = nec_data, 
+                  x_var = "x", 
+                  y_var = "y", 
+                  model = "decline")
\end{CodeInput}
\end{CodeChunk}

Only three arguments must be supplied, including: \texttt{data}---a
\texttt{data.frame} containing the data to use for the model;
\texttt{x\_var}---a \texttt{character} indicating the column heading
containing the concentration (x) variable; and \texttt{y\_var}---a
\texttt{character} indicating the column heading containing the response
(y) variable. The next argument \texttt{model} is a \texttt{character}
indicating the desired model (or model-set, see details below) and
currently defaults to ``all'' models available. A large range of
arguments to be passed to \pkg{brms} or model weighting via \texttt{loo}
can be specified manually by the user as required for more advanced
users.

While the statistical distribution of the response variable (y) can be
specified directly, \pkg{bayesnec} will automatically try and predict
the most appropriate distribution (family) to use based on the
characteristics of the provided data, as well as build the relevant
model priors (see more details below) and initial values for the
\pkg{brms} model. Initial values are selected such that they yield
predicted values in the range of the observed response (\texttt{y\_var})
data.

If not supplied, the appropriate family to use will be determined based
on the characteristics of the input data. This includes evaluation of
class (integer or numeric) as well as the observed range via the
function \texttt{check\_data}. Family predictions include all of the
available families (see above) except Negative-binomial and
Beta-binomial because these require knowledge on whether the data are
overdispersed, which are not assessed prior to model fitting. By default
\pkg{bayesnec} will use the default link function for the predicted
family. Note that in \texttt{bnec} the default link for the Gamma family
has been set to \texttt{log}. If other link functions are required for
any of the implemented families, such as \texttt{identity} in the case
of a Beta family for example, this will need to be specified manually by
adding the argument \texttt{family\ =\ Beta(link\ =\ "identity")} in the
\texttt{bnec} call.

Data that are numeric and scaled from 0 to 1, or 0 to \(\infty\) are
assumed to be Beta- and Gamma-distributed respectively. Because both
distributions are naturally truncated at 0, if necessary a value that is
0 is increased by 1/10\textsuperscript{th} of the next smallest non-zero
value. Similarly, because Beta is also truncated at 1, ones are
substituted with 0.999. Data scaled from or -\(\infty\) to \(\infty\)
are modelled using a Gaussian distribution.

If \texttt{y\_var} data are integers and a \texttt{trials\_var} argument
is supplied (must also be integer) a Binomial distribution is assumed.
If no \texttt{trials\_var} argument is supplied the \texttt{y\_var} data
are assumed to be Poisson if they are integers. Similar checks are used
to assign a likely family to the \texttt{x\_var} (concentration) data.
In this case of \texttt{x\_var} data, the assigned family only
influences the priors on \texttt{x\_var}-based model parameters
(\emph{NEC}), which can be overridden by setting manual priors (see more
information on priors below).

Specific models can be fit directly using \texttt{bnec}, in which case
an object of class \texttt{bayesnecfit} is returned. \pkg{bayesnec}
includes a set of \texttt{bayesnecfit} methods for predicting, plotting
and extracting effect concentrations and related threshold values.
Alternatively it is possible to fit a custom model-set, specific
model-set or all of the available models. The default in \pkg{bayesnec}
is to use \texttt{model\ =\ "all"} which fits all of the available
models.

When multiple models are fitted using \texttt{bnec}, an object of class
\texttt{bayesmanecfit} is returned that includes a model weighted
estimate of predicted posterior values. \pkg{bayesnec} also includes a
set of \texttt{bayesmanecfit} methods for predicting, plotting and
extracting effect concentrations and related threshold values that
return model weighted estimates.

Multi-model inference can be useful where there are a range of plausible
models that could be used \citep{Burnham2002} and has been recently
adopted in ecotoxicology for Species Sensitivity Distribution (SSD)
model inference \citep{Thorley2018, fox2020, Dalgarno}. The approach may
have considerable value in C-R modelling because there is often no a
priori knowledge of the functional form that the response relationship
should take. In this case, model averaging can be a useful way of
allowing the data to drive the model selection process, with weights
proportional to how well the individual models fit the data.
Well-fitting models will have high weights, dominating the model
averaged outcome. Conversely, poorly fitting models will have very low
model weights and will therefore have little influence on the outcome.
Where multiple models fit the data equally well, these can equally
influence the outcome, and the resultant posterior predictions reflect
that model uncertainty. It is possible to specify the ``stacking''
method \citep{Yao2018} for model weights if desired (through the
argument \texttt{loo\_controls}) which aims to minimise prediction
error. We do not currently recommend using stacking weights given the
typical sample sizes associated with most C-R experiments, and because
the main motivation for model averaging within the \pkg{bayesnec}
package is to properly capture model uncertainty rather than reduce
prediction error. By default \pkg{bayesnec} uses the ``pseudobma''
method using the Bayesian bootstrap through \texttt{loo\_model\_weights}
\citep{vehtari2020, vehtari2017}. These are analogous to the way model
weights are generated using AIC or AICc \citep{Burnham2002}, although
more research on their behaviour and suitability for model averaging in
the current non-linear modelling setting is warranted.

\hypertarget{model-diagnostics}{%
\subsubsection{Model diagnostics}\label{model-diagnostics}}

A range of tools are available to assess model fit, including an
estimate of overdispersion (for relevant families), an extension of the
\pkg{brms} \texttt{rhat} function that can be applied to both
\texttt{bayesnecfit} and \texttt{bayesmanecfit} model objects, and a
function \texttt{check\_chains} that can be used to visually assess
chain mixing and stability.

All diagnostic functions available in \pkg{brms} and \pkg{rstan} can be
used on the underlying \texttt{brm} model fit by extracting the fitted
\pkg{brms} model from the \texttt{bayenecfit} or \texttt{bayesmanecfit}
model object. For example, we can use the default \pkg{brms} plotting
method to obtain a diagnostic plot of the individual fit of the
\textbf{nec4param} model using:

\begin{CodeChunk}
\begin{CodeInput}
R> plot(exmp_fit$mod_fits$nec4param$fit)
\end{CodeInput}
\begin{figure}

{\centering \includegraphics{article_files/figure-latex/brms-plot-1} 

}

\caption[Default \pkg{brms} plot of the \texttt{nec4param} model showing the posterior probability densities and chain mixing for each of the included parameters]{Default \pkg{brms} plot of the \texttt{nec4param} model showing the posterior probability densities and chain mixing for each of the included parameters.}\label{fig:brms-plot}
\end{figure}
\end{CodeChunk}

\newpage

which yields a plot of the posterior densities as well as trace plots of
chains for each parameter in the specified model
(\autoref{fig:brms-plot}).

The default number of total iterations in \pkg{bayesnec} is 10,000 per
chain, with 9,000 of these used as warm-up (or burn-in) across 4 chains.
If the \texttt{bnec} call returns \pkg{brms} warning messages the number
of iterations and warm-up samples can be adjusted through arguments
\texttt{iter} and \texttt{warmup}. A range of other arguments can be
further adjusted to improve convergence, see the rich set of
\href{https://github.com/paul-buerkner/brms}{Resources} available for
the \pkg{brms} package for further information.

Several helper functions have been included that allow the user to add
or drop models from a \texttt{bayesmanecfit} object, or change the model
weighting method (\texttt{amend}); extract a single or subset of models
from the \texttt{bayesmanecfit} object (\texttt{pull\_out}); and examine
the priors used for model fitting (\texttt{pull\_prior},
\texttt{sample\_priors} and \texttt{check\_priors}).

\hypertarget{model-inference}{%
\subsubsection{Model inference}\label{model-inference}}

\pkg{bayesnec} includes a summary method for both \texttt{bayesnecfit}
and \texttt{bayesmanecfit} model objects, providing the usual summary of
model parameters and any relevant model fit statistics as returned in
the underlying \texttt{brm} model fits. This includes a list of fitted
models, their respective model weights, and a model-averaged
\emph{NEC}---which is reported with a warning when it contains NSEC
values (see more information below). A warning message also indicates
that the \texttt{ecxll5} model may have convergence issues according to
the default \pkg{brms} \(\widehat{R}\) criteria:

\begin{CodeChunk}
\begin{CodeInput}
R> summary(exmp_fit)
\end{CodeInput}
\begin{CodeOutput}
Object of class bayesmanecfit containing the following non-linear models:
  -  nec4param
  -  neclin
  -  ecxlin
  -  ecx4param
  -  ecxwb1
  -  ecxwb2
  -  ecxll5
  -  ecxll4

Distribution family: beta
Number of posterior draws per model:  4000

Model weights (Method: pseudobma_bb_weights):
             waic   wi
nec4param -333.23 0.57
neclin    -332.67 0.43
ecxlin    -188.35 0.00
ecx4param -302.59 0.00
ecxwb1    -293.75 0.00
ecxwb2    -316.90 0.00
ecxll5    -319.41 0.00
ecxll4    -301.97 0.00


Summary of weighted NEC posterior estimates:
NB: Model set contains the ECX models: ecxlin;ecx4param;ecxwb1;ecxwb2;ecxll5;ecxll4; weighted NEC estimates include NSEC surrogates for NEC
    Estimate Q2.5 Q97.5
NEC     1.41 1.29  1.48
\end{CodeOutput}
\end{CodeChunk}

Base R (\texttt{plot}) and ggplot2 (\texttt{ggbnec}) plotting methods,
as well as predict methods have also been developed for both
\texttt{bayesnecfit} and \texttt{bayesmanecfit} model classes. In
addition, there are method-based functions for extracting \emph{ECx}
(\texttt{ecx}), \emph{NEC} (\texttt{nec}) and \emph{NSEC}
(\texttt{nsec}) threshold values. In all cases the posterior samples
that underpin these functions are achieved through
\texttt{posterior\_epred} from the \pkg{brms} package. An example base
plot of a \texttt{bayesmanecfit} model fit can be seen in
\autoref{fig:base-plot}.

\begin{CodeChunk}
\begin{CodeInput}
R> plot(exmp_fit)
\end{CodeInput}
\begin{figure}

{\centering \includegraphics{article_files/figure-latex/base-plot-1} 

}

\caption[Base plot of the example fit model averaged curve]{Base plot of the example fit model averaged curve. The solid black line is the fitted median of the posterior prediction, dashed black lines are the 95\% credible intervals, and the red vertical lines show the estimated \textit{NEC} value.}\label{fig:base-plot}
\end{figure}
\end{CodeChunk}

By default the plot shows the fitted posterior curve with 95\% credible
intervals, along with an estimate of the \(\eta = \text{NEC}\) value.
Please see the
\href{https://open-aims.github.io/bayesnec/articles/}{vignettes} for
more examples using \pkg{bayesnec} models for inference.

\subsection[Models in bayesnec]{Models in
\pkg{bayesnec}}\label{models-in}

The argument \texttt{model} in the function \texttt{bnec} is a character
string indicating the name(s) of the desired model. If a recognised
model name is provided, a single model of the specified type is fit, and
\texttt{bnec} returns a model object of class \texttt{bayesnecfit}. If a
vector of two or more of the available models are supplied,
\texttt{bnec} returns a model object of class \texttt{bayesmanecfit}
containing Bayesian model averaged predictions for the supplied models,
providing they were successfully fitted.

The \texttt{model} argument may also be one of: \texttt{all}, meaning
all of the available models will be fit; \texttt{ecx}, meaning only
models excluding the \(\eta = \text{NEC}\) step parameter will be fit;
or \texttt{nec}, meaning only models with the \(\eta = \text{NEC}\) step
parameter (see below
\protect\hyperlink{parameter-definitions}{\textit{Parameter definitions}})
will be fit. There are a range of other pre-defined model groups
available. The full list of currently implemented model groups can be
seen using the function \texttt{models()}.

Please see the
\href{https://open-aims.github.io/bayesnec/articles/example2b.html}{Model
details} vignette or \texttt{?models("all")} for more information on all
the models available in \pkg{bayesnec} and their specific formulation.

\hypertarget{parameter-definitions}{%
\subsubsection{Parameter definitions}\label{parameter-definitions}}

Where possible we have aimed for consistency in the interpretable
meaning of the individual parameters across models. Across the currently
implemented model-sets, models contain from two (basic linear or
exponential decay, see
e.g.~\texttt{models("all"){[}c("ecxlin",\ "ecxexp"){]}} to five possible
parameters
(e.g.~\texttt{models("all"){[}c("nechorme4",\ "ecxhormebc5"){]}}),
including:

\begin{itemize}
\item
  \(\tau = \text{top}\), usually interpretable as either the y-intercept
  or the upper plateau representing the mean concentration of the
  response at zero concentration;
\item
  \(\eta = \text{NEC}\), the no-effect-concentration value (the x
  concentration value at which the breakpoint in the regression is
  estimated, see
  \protect\hyperlink{model-types-for-nec-and-ecx-estimation}{\textit{Model types for NEC and ECx estimation}}
  and \citet{Fox2010} for more details on parameter based \emph{NEC}
  estimation);
\item
  \(\beta = \text{beta}\), generally the exponential decay rate of
  response, either from 0 concentration or from the estimated \(\eta\)
  value, with the exception of the \texttt{neclinhorme} model where it
  represents a linear decay from \(\eta\) because slope (\(\alpha\)) is
  required for the linear increase;
\item
  \(\delta = \text{bottom}\), representing the lower plateau for the
  response at infinite concentration;
\item
  \(\alpha = \text{slope}\), the linear decay rate in the models
  \texttt{neclin} and \texttt{ecxlin}, or the linear increase rate prior
  to \(\eta\) for all hormesis models;
\item
  \(\text{ec50}\) notionally the 50\% effect concentration but may be
  influenced by scaling and should therefore not be strictly
  interpreted; and
\item
  \(\epsilon = \text{d}\), the exponent in the \texttt{ecxsigm} and
  \texttt{necisgm} models.
\end{itemize}

In addition to the model parameters, all \texttt{nec...} models have a
step function used to define the breakpoint in the regression, which can
be defined as:

\[
f(x_i, \eta) = \begin{cases} 
      0, & x_i - \eta < 0 \\
      1, & x_i - \eta \geq 0 \\
   \end{cases}
\]

\hypertarget{model-types-for-nec-and-ecx-estimation}{%
\subsubsection{Model types for NEC and ECx
estimation}\label{model-types-for-nec-and-ecx-estimation}}

All models provide an estimate of the no-effect-concentration
(\emph{NEC}). For model types with ``nec'' as a prefix, the \emph{NEC}
is directly estimated as parameter \(\eta = \text{NEC}\) in the model,
as per \citet{Fox2010}. Models with ``ecx'' as a prefix are continuous
curve models, typically used for extracting \emph{ECx} values from C-R
data. In this instance the \emph{NEC} reported is actually the
no-significant-effect-concentration (\emph{NSEC}), defined as the
concentration at which the predicted response is now ``significantly''
lower than the values observed at the lowest treatment concentration
(typically the `control'). The desired level of ``significance'' can be
controlled by the user through the argument \texttt{sig\_val}, which is
passed as a probability to \texttt{quantile} and used to calculate the
lower bound of the posterior predictions at the lowest treatment
concentration. The default value for \texttt{sig\_val} is 0.01, which is
analogous to an alpha value (Type-I error rate) of 0.01 for a one-sided
test of significance. See \texttt{?nsec} for more details. We currently
recommend only using the ``nec'' model-set for estimation of \emph{NEC}
values, as the \emph{NSEC} concept has yet to be formally peer-reviewed,
and likely suffers from at least some of the same issues as the
No-observed-effect-concentration
\citep[\emph{NOEC},][]{Warne2008a, Fox2008}.

\emph{ECx} estimates can be equally validly obtained from both ``nec''
and ``ecx'' models. \emph{ECx} estimates will usually be lower (more
conservative) for ``ecx'' models fitted to the same data as ``nec''
models (see the
\href{https://open-aims.github.io/bayesnec/articles/example4.html}{Comparing
posterior predictions} vignette for an example. However, we recommend
using ``all'' models where \emph{ECx} estimation is required because
``nec'' models can fit some datasets better than ``ecx'' models and the
model averaging approach will place the greatest weight for the outcome
that best fits the supplied data. This approach will yield \emph{ECx}
estimates that are the most representative of the underlying
relationship in the dataset.

There is ambiguity in the definition of \emph{ECx} estimates from
hormesis models (these allow an initial increase in the response
\citep[see][]{Mattson2008} and include models with the character string
\texttt{horme} in their name), as well as those that have no natural
lower bound on the scale of the response (models with the string
\texttt{lin} in their name, in the case of Gaussian response data). For
this reason, the \texttt{ecx} function has arguments
\texttt{hormesis\_def} and \texttt{type}, both character vectors
indicating the desired behaviour. For \texttt{hormesis\_def\ =\ "max"},
\emph{ECx} values are calculated as a decline from the maximum estimates
(i.e.~the peak at \(\eta = \text{NEC}\)); and
\texttt{hormesis\_def\ =\ "control"} (the default) indicates \emph{ECx}
values should be calculated relative to the control, which is assumed to
be the lowest observed concentration. For \texttt{type\ =\ "relative"},
\emph{ECx} is calculated as the percentage decrease from the maximum
predicted value of the response (\(\tau = \text{top}\)) to the minimum
predicted value of the response (ie, `relative' to the observed data).
For \texttt{type\ =\ "absolute"} (the default) \emph{ECx} is calculated
as the percentage decrease from the maximum value of the response
(\(\tau = \text{top}\)) to 0 (or \(\delta = \text{bottom}\) for models
with that parameter). For \texttt{type\ =\ "direct"}, a direct
interpolation of y on x is obtained.

\hypertarget{model-suitability-for-response-types}{%
\subsubsection{Model suitability for response
types}\label{model-suitability-for-response-types}}

Models that have an exponential decay (most models with parameter
\(\beta = \text{beta}\)) with no \(\delta = \text{bottom}\) parameter
are 0 bounded and are not suitable for the Gaussian family, or any
family modelled using a logit or log link because they cannot generate
predictions of negative y (response). Conversely models with a linear
decay (containing the string \texttt{lin} in their name) are not
suitable for modelling families that are 0 bounded (Gamma, Poisson,
Negative-binomial, Beta, Binomial, Beta-binomial) using an identity
link. These restrictions do not need to be controlled by the user as a
call to \texttt{bnec} with \texttt{models\ =\ "all"} will simply exclude
inappropriate models, albeit with a message.

Strictly speaking, models with a linear hormesis increase are not
suitable for modelling data from the Binomial, Beta and Beta-binomial
families, however they are currently allowed in \pkg{bayesnec}, with a
reasonable fit achieved through a combination of the appropriate family
being applied to the response, and the \pkg{bayesnec}
\texttt{make\_inits} function that ensures initial values passed to
\pkg{brms} yield response values within the range of the observed
\texttt{y\_var} data.

\hypertarget{priors-on-model-parameters}{%
\subsection{Priors on model
parameters}\label{priors-on-model-parameters}}

In Bayesian inference, model parameters and their inherent uncertainty
are estimated as statistical probability distributions. This is achieved
by combining an a-priori understanding of each parameter's probability
density function (the `priors') with the likelihood of the observed
information (data) given the model parameters, to yield a so-called
posterior probability distribution. Regardless of whether this is done
mathematically via Bayes' theorem or through Monte Carlo simulation, to
carry out a Bayesian analysis the prior probability densities must be
defined. Sometimes there may be substantial prior knowledge, for example
when pilot data or data from a previous experiment exist for a given
response curve. In this case the prior probability distribution may be
quite narrow (highly ``informative'') and can have a strong influence on
the posterior, especially when subsequent data are scarce or highly
variable. In our experience however, such prior knowledge is generally
the exception. Where no quantitative prior information exists, it is
common in Bayesian statistics to use ``vague'' or ``weakly'' informative
priors. The use of ``vague'', ``diffuse'', ``flat'' or otherwise
so-called ``uninformative'' priors is no longer recommended
\citep{Banner2020}. Such priors generally form the default for many
Bayesian packages, and are often used in practice without critical
thought or evaluation, possibly as a result of fear of being too
`subjective' \citep{Banner2020}.

Considerable thought has gone into development of an algorithm
(\texttt{define\_prior}) to build ``weakly'' informative priors for
fitting models in \pkg{bayesnec}. The priors are ``weakly'' informative
in that in addition to specifying the relevant statistical family that
appropriately captures the parameters likely theoretical statistical
distribution, we also use information contained within the observed data
to centre the probability density near the most likely parameter space
and/or constrain priors to sensible bounds. These weakly informative
priors are used to help constrain the underlying routines so that they
are less likely to consider what the researcher would deem highly
improbable estimates, that also cause the routines to become unstable.
Weakly informative priors can be particularly helpful in complex
non-linear modelling to ensure reliable convergence. These types of
priors specify the general shape and bounds of the expected probability
distribution for a given parameter, whilst remaining sufficiently broad
so as not to influence the parameter's estimated posterior distribution
(given a reasonable amount of observed data). In this sense
appropriately weak priors should yield analytical outcomes that share
the same level of \emph{objectivity} as equivalent frequentist
approaches, whilst yielding robust parameter estimates with
probabilistically interpretable uncertainty bounds. To ensure a
\pkg{bayesnec} analysis retains the same level of objectivity as an
equivalent frequentist approach we recommend using the function
\texttt{check\_priors} to compare the prior and posterior probability
densities and check that the priors used by \pkg{bayesnec} are sensible
and are not exerting an undesirable influence over the analysis.

\subsubsection[bayesnec's define_prior algorithm]{\pkg{bayesnec}'s
\texttt{define\_prior} algorithm}\label{s-define_prior-algorithm}

Priors are constructed in \pkg{bayesnec} for each parameter of each
model being fitted based on the characteristics of either the input
\texttt{x\_var} or \texttt{y\_var} data, depending on which is relevant
to the specific parameter scaling. In the case of parameters that scale
with \texttt{y\_var} (the response), priors are constructed based on the
relevant link scaling, whether that be identity, the default (for that
family), or user specified link function for a specific family. The
priors are constructed by \texttt{bnec} internally, calling the function
\texttt{define\_prior}, which takes the arguments \texttt{model},
\texttt{family} (including the relevant link function),
\texttt{predictor} (\texttt{x\_var} data), and \texttt{response}
(\texttt{y\_var} data).

\hypertarget{priors-for-response-y_var-scaled-parameters}{%
\paragraph{\texorpdfstring{Priors for response (\texttt{y\_var}) scaled
parameters}{Priors for response (y\_var) scaled parameters}}\label{priors-for-response-y_var-scaled-parameters}}

Only the parameters \(\tau = \text{top}\) and \(\delta = \text{bottom}\)
scale specifically with the response (\texttt{y\_var} data) family. For
Gaussian \texttt{y\_var} data (or any \texttt{y\_var} data for which the
link ensures valid values of the response can take from -\(\infty\) to
\(\infty\), including \texttt{log} and \texttt{logit}) priors are
Gaussian with a standard deviation of 2.5 and a mean set at the
90\textsuperscript{th} and 10\textsuperscript{th} quantiles for
\(\tau = \text{top}\) and \(\delta = \text{bottom}\) respectively. In
this way \pkg{bayesnec} attempts to construct a prior that scales
appropriately with the observed data, with greatest density near the
most likely region of the response for the \(\tau = \text{top}\) and
\(\delta = \text{bottom}\) parameters, whilst remaining broad enough to
have little influence on each parameter's posterior density.

For Poisson-, Negative-binomial- and Gamma-distributed \texttt{y\_var}
data, the response is bounded by 0 and therefore Gaussian priors are
unsuitable. Instead we use Gamma priors, with a mean scaled to
correspond to the 75\textsuperscript{th} and 25\textsuperscript{th}
quantiles for \(\tau = \text{top}\) and \(\delta = \text{bottom}\)
respectively. The mean (\(\mu\)) is linked mathematically to the shape
(s) and rate parameters (r) by the equation \[ \mu = s * (1/r) \] with
the shape parameter being set to 2 by default.

For the Binomial, Beta, and Beta-binomial families, estimates for
\(\tau = \text{top}\) and \(\delta = \text{bottom}\) must necessarily be
constrained between 0 and 1 when modelled on the identity link. Because
of this constraint there is no need to adjust scaling based on the
response. In this case \pkg{bayesnec} uses \texttt{beta(5,\ 1)} and
\texttt{beta(1,\ 5)} priors to provide a broad density centred across
the upper and lower 0 to 1 range for the \(\tau = \text{top}\) and
\(\delta = \text{bottom}\) parameters respectively.

\hypertarget{priors-for-predictor-x_var-scaled-parameters}{%
\paragraph{\texorpdfstring{Priors for predictor (\texttt{x\_var}) scaled
parameters}{Priors for predictor (x\_var) scaled parameters}}\label{priors-for-predictor-x_var-scaled-parameters}}

The parameters \(\eta = \text{NEC}\) and \(\eta = \text{ec50}\) scale
with respect to the predictor (\texttt{x\_var} data), because both of
these are estimated in units of the predictor (\texttt{x\_var}, usually
concentration). To stabilise model fitting, the \(\eta = \text{NEC}\)
and \(\eta = \text{ec50}\) parameters are bounded to the upper and lower
observed range in the predictor, under the assumption that the range of
concentrations in the experiment were sufficient to cover the full range
of the response outcomes. The priors used reflect the characteristics of
the observed data that are used to predict the appropriate family. If
the \texttt{x\_var} data are bounded to 0 and \textgreater1 a Gamma
prior is used, with maximum density (\(\mu\), see above) at the median
value of the predictor, and a shape parameter of 5. If the
\texttt{x\_var} data are bounded to 0 and 1 a \texttt{beta(2,\ 2)} prior
is used. For \texttt{x\_var} data ranging from -\(\infty\) to
\(\infty\), a Gaussian prior is used, with a mean set at the median of
the \texttt{x\_var} values and a standard deviation of 2.5.

\hypertarget{priors-for-other-parameters}{%
\paragraph{Priors for other
parameters}\label{priors-for-other-parameters}}

For the parameters \(\beta = \text{beta}\), \(\alpha = \text{slope}\)
and \(\epsilon = \text{d}\) we first ensured any relevant
transformations in the model formula such that theoretical values with
the range -\(\infty\) to \(\infty\) are allowable, and a
\texttt{normal(0,\ 2)} prior is used. For example in the
\texttt{nec3param} model \(\beta = \text{beta}\) is an exponential decay
parameter, which must by definition be bounded to 0 and \(\infty\).
Calling \texttt{exp(beta)} in the model formula ensures the exponent
meets these requirements. Note also that a mean of 0 and standard
deviation of 2 represents a relatively broad prior on this exponential
scaling, so this is still generally a weakly informative prior in
practice.

\hypertarget{user-specified-priors}{%
\subsubsection{User specified priors}\label{user-specified-priors}}

There may be situations where the default \pkg{bayesnec} priors do not
behave as desired, or the user wants to provide informative priors. For
example, the default priors may be too informative, yielding
unreasonably tight confidence bands (although this is only likely where
there are few data or unique values of the \texttt{x\_var} data).
Conversely, priors may be too vague, leading to poor model convergence.
Alternatively, the default priors may be of the wrong statistical family
if there was insufficient information in the provided data for
\pkg{bayesnec} to correctly predict the appropriate ones to use. The
priors used in the default model fit can be extracted using
\texttt{pull\_prior}, and a sample or plot of prior values can be
obtained from the individual \pkg{brms} model fits through the function
\texttt{sample\_priors} which samples directly from the \texttt{prior}
element in the \texttt{brm} model fit (see \autoref{fig:sample-prior}).

\begin{CodeChunk}
\begin{CodeInput}
R> sample_priors(exmp_fit$mod_fits$nec4param$fit$prior)
\end{CodeInput}
\begin{figure}

{\centering \includegraphics{article_files/figure-latex/sample-prior-1} 

}

\caption[Frequency histograms of samples of the default priors used by bnec for fitting the \texttt{nec4param} model to the example data]{Frequency histograms of samples of the default priors used by bnec for fitting the \texttt{nec4param} model to the example data.}\label{fig:sample-prior}
\end{figure}
\end{CodeChunk}

We can also use the function \texttt{check\_priors} (based on the
\texttt{hypothesis} function of \pkg{brms}) to assess how the posterior
probability density for each parameter differs from that of the prior.
Here we show the prior and posterior probability densities for the
parameters in the \texttt{nec4param} model, extracted from our example
fit (see \autoref{fig:check-priorsingle}). There is also a class
\texttt{bayesmanecfit} method that can be used to sequentially view all
plots in a \texttt{bnec} call with multiple models, or write to a pdf as
in \texttt{check\_chains}.

This can be done for a single model using:

\begin{CodeChunk}
\begin{CodeInput}
R> exmp_fit_nec4param <- pull_out(exmp_fit, model = "nec4param")
R> check_priors(exmp_fit_nec4param)
\end{CodeInput}
\begin{figure}

{\centering \includegraphics{article_files/figure-latex/check-priorsingle-1} 

}

\caption[A comparison of the prior and posterior parameter probability densities for the \texttt{nec4param} model fit to the example data]{A comparison of the prior and posterior parameter probability densities for the \texttt{nec4param} model fit to the example data.}\label{fig:check-priorsingle}
\end{figure}
\end{CodeChunk}

or for all models, writing to a pdf file named Check\_priors\_plots.pdf:

\begin{CodeChunk}
\begin{CodeInput}
R> check_priors(exmp_fit, filename = "Check_priors_plots")
\end{CodeInput}
\end{CodeChunk}

\hypertarget{model-comparison}{%
\subsection{Model comparison}\label{model-comparison}}

With \pkg{bayesnec} we have included a function
(\texttt{compare\_posterior}) that allows bootstrapped comparisons of
posterior predictions. This function allows the user to fit several
different \texttt{bnec} model fits and compare differences in the
posterior predictions. Comparisons can be made across the model fits for
individual endpoint estimates (e.g.~\emph{NEC}, \emph{NSEC} or
\emph{ECx}) or across a range of predictor (x) values. Usage is
demonstrated in the relevant
\href{https://open-aims.github.io/bayesnec/articles/example4.html}{vignette}
by comparing different types of models and model-sets using a single
dataset. However, the intent of this function is to allow comparison
across different datasets that might represent, for example, different
levels of a fixed factor covariate. For example, this function has been
used to compare toxicity of herbicides across three different climate
scenarios, to examine the cumulative impacts of pesticides and global
warming on corals \citep{flores2021}.

At this time \texttt{bnec} does not allow for an inclusion of an
interaction with a fixed factor. Including an interaction term within
each of the non-linear models implemented in \pkg{bayesnec} is
relatively straightforward, and may be introduced in future releases.
However, in many cases the functional form of the response may change
with different levels of a given factor. The substantial complexity of
defining all possible non-linear model combinations at each factor level
means it unlikely this could be feasibly implemented in \pkg{bayesnec}
in the short term. In the meantime the greatest flexibility in the
functional form of individual model fits can be readily obtained using
models fitted independently to data within each factor level.

\hypertarget{hierarchical-effects}{%
\subsection{Hierarchical effects}\label{hierarchical-effects}}

Most ecotoxicological and toxicology experiments include a range of
grouping elements, such as tanks, vials or batches of samples that
contain multiple measurements that cannot be considered strictly
independent (a.k.a. they are pseudo-replicates). To avoid criticism
around potential issues with pseudo-replication, it is often the
practice for ecotoxicologists to pool such observations and carry out
modelling using, for example, the group mean. Where the number of within
group observations varies substantially across groups, this will have
the undesirable effect of equally weighting the group means even though
some may be based on far fewer observations than others. In addition
there are often instances of ecotoxicology data from multiple
experiments or other grouping factors within an experiment (such as
genotype) that cover the full range of x concentrations that cannot be
averaged prior to modelling, resulting in the ecotoxicologist either
ignoring the potential non-independence, or fitting many independent
datasets and subsequently needing to aggregate the endpoint estimates.
Carrying out multiple fits on separate datasets is undesirable because
each fit is based on fewer data and will have greater uncertainty.

The current version of \pkg{bayesnec} harnesses the powerful mixed
(hierarchical) modelling capabilities of \pkg{brms} for accommodating
hierarchical designs and other forms of non-independence. This is
achieved by allowing a list of grouping terms to be passed through the
argument \emph{random}, which are then used to update the \pkg{brms}
model formula. Hierarchical effects can be in the form of an offset,
which effectively allows different mean response levels across groups,
and is achieved through specifying the term \texttt{ost} in the named
list. Hierarchical effects can also be added to any or all of the
non-linear parameters in the model. Note that implementing hierarchical
effects in a non-linear modelling setting is non-trivial and
considerable thought and testing should go into selecting an appropriate
hierarchical structure.

\hypertarget{discussion}{%
\section{Discussion}\label{discussion}}

In order to be accessible to a broad community of varying statistical
capabilities, we have simplified fitting a \pkg{bayesnec} model as much
as possible, whilst retaining the ability to modify a wide range of
arguments as necessary. Where possible we have tried to set default
values to align with those in \pkg{brms}. Wherever we deviate, this is
generally towards being more conservative and/or we have clearly
explained our reasoning. Specific examples include: 1) \texttt{iter},
which we increased from the \pkg{brms} default of 2,000 to 10,000 as we
found that a higher number of iterations are generally required for
these non-linear models; 2) the use of pseudo-BMA rather than stacking
weights to better reflect \pkg{bayesnec}'s goals with respect to model
averaging (discussed in more detail above); and 3) the use of
\texttt{pointwise\ =\ TRUE} (where possible) and
\texttt{sample\_prior\ =\ "yes"} to avoid excessive R crashes when used
in the Windows operating system and allow the use of the
\texttt{hypothesis} function respectively. We welcome constructive
criticism of our selections and users must expect that default settings
may change accordingly in later releases.

We have made considerable effort to ensure that \pkg{bayesnec} makes a
sensible prediction for the appropriate family, constructs appropriate
weakly informative priors, and generates sensible initial values.
However, this is a difficult task across such a broad range of
non-linear models, and across the potential range of ecotoxicological
data that may be used. The user must interrogate their model fits using
the wide array of helper functions, and use their own judgement
regarding the appropriateness of model inferences for their own
application. Of particular importance are examination of model fit
statistics through the \texttt{summary} and \texttt{rhat} methods,
visual inspection of all model fits in \texttt{bayesmanecfit} objects
(via \texttt{plot(...,\ all\_models\ =\ TRUE)} and
\texttt{check\_chains(...,\ all\_models\ =\ TRUE)}) and an assessment of
the posterior versus prior probability densities to ensure default
priors are appropriate (using \texttt{check\_priors}).

The model averaging approach implemented in \pkg{bayesnec} is widely
used in a range of settings \citep[in ecology for example, see][ for a
thorough review]{Dormann2018}. However, model averaging is relatively
new to ecotoxicology \citep[but see, for
example,][]{Shao2014, Thorley2018, fox2020, Wheeler2009}. In
\pkg{bayesnec} we have implemented a broad range of potential models,
and the default behaviour is to fit them all, although we discuss above
situations where this is clearly not recommended (for example, in the
estimation of \emph{NEC}). More research is required to understand how
model-set selection influences model inference. While some studies
suggest using a broad range of models may be optimal
\citep{Wheeler2009}, others indicate that including multiple models of
similar shape may overweight the representation of that shape in model
averaged predictions \citep{fox2020}. In addition, it is important to
understand that when models are added or removed from the model-set,
this can sometimes have a substantial influence on model predictions
(potentially changing estimated \emph{ECx} values, for example). As the
model-set in \pkg{bayesnec} may change through time it is important to
keep a record of the models that were actually fitted in a given
analysis, in the event it is necessary to reproduce a set of results. A
potentially better strategy is to build a
\href{https://docs.docker.com/get-docker/}{Docker} container, an
emerging approach representing one strategy towards overcoming the
reproducibility crisis \citep{Baker2016}. Considerations of analytical
reproducibility are particularly relevant to C-R modelling, where the
model outcomes can often have far reaching management implications.

\hypertarget{future-directions}{%
\section{Future directions}\label{future-directions}}

The \pkg{bayesnec} package is a work in progress, and we welcome
suggestions and feedback that will improve the package performance and
function. Our goal is to make \pkg{bayesnec} as user friendly as
possible, and capable of dealing with most real world C-R modelling
applications in the hope that Bayesian statistics will become more
widely used in applied risk assessment. Please submit requests through
the package \href{https://github.com/open-AIMS/bayesnec/issues}{Issues}
on GitHub. Some suggested future enhancements include:

\begin{itemize}
\item
  The addition of other custom families, such as the Tweedie
  distribution and ordered beta model. Currently \pkg{bayesnec}
  implements adjustments away from 0 (Gamma, Beta) or 1 (Beta) as a
  strategy for allowing modelling with these types of data using the
  closest most convenient statistical distribution. There are no readily
  available distributions able to model data that includes 0 and 1 on
  the continuous scale in \pkg{brms} and \pkg{bayesnec} currently does 0
  and 1 adjustments followed by modelling using a Beta distribution. The
  ordered beta model has been suggested as a better method for modelling
  continuous data with lower an upper bounds (see \citep{Kubinec}) that
  could be readily implemented in the \pkg{brms} customs families
  framework. For data that are 0 to \(\infty\) on the continuous scale
  the Tweedie distribution may prove a much better option than the
  current zero-bounded Gamma, and has been used extensively in fisheries
  research for biomass data \citep{Shono2008}. As this family is not
  currently available in \pkg{brms} this would also need to be
  implemented as a custom family, which for the Tweedie is not trivial.
\item
  A hypothesis method for testing against toxicity thresholds. The
  \pkg{brms} package includes a \texttt{hypothesis} function that allows
  for testing parameter estimates against specified criteria. This is
  used in \pkg{bayesnec} in the \texttt{check\_prior} function, which is
  a wrapper that examines the deviation of each parameter in the given
  model relative to 0 as a means of generating posterior and prior
  probability density plots for comparison. However, an additional
  wrapper function could be developed that allows toxicity to be
  assessed, as measured through \emph{NEC}, or \emph{ECx} for example,
  against a required pre-defined threshold. Such a feature may be useful
  where toxicity testing is used as a trigger in risk management
  \citep[for example, using whole-effluent-toxicity (WET)
  testing,][]{Karman2019}.
\end{itemize}

\hypertarget{acknowledgements}{%
\section{Acknowledgements}\label{acknowledgements}}

The development of \pkg{bayesnec} was supported by an AIMS internal
grant. Usage, testing and functionality of both the \pkg{jagsNEC} and
\pkg{bayesnec} packages were substantially aided through input from
Joost van Dam, Andrew Negri, Florita Flores, Heidi Luter, Marie Thomas
and Mikaela Nordborg. Florita Flores and Murray Logan provided valuable
comments on the manuscript text. Ron Jones provided technical computing
support.

\renewcommand\refname{References}
\bibliography{refs.bib}


\end{document}
