\documentclass[10pt,a4paper,onecolumn]{article}
\usepackage{marginnote}
\usepackage{graphicx}
\usepackage{xcolor}
\usepackage{authblk,etoolbox}
\usepackage{titlesec}
\usepackage{calc}
\usepackage{tikz}
\usepackage{hyperref}
\hypersetup{colorlinks,breaklinks,
            urlcolor=[rgb]{0.0, 0.5, 1.0},
            linkcolor=[rgb]{0.0, 0.5, 1.0}}
\usepackage{caption}
\usepackage{tcolorbox}
\usepackage{amssymb,amsmath}
\usepackage{ifxetex,ifluatex}
\usepackage{seqsplit}
\usepackage{fixltx2e} % provides \textsubscript
\usepackage[
  backend=biber,
%  style=alphabetic,
%  citestyle=numeric
]{biblatex}
\bibliography{paper.bib}


% --- Page layout -------------------------------------------------------------
\usepackage[top=3.5cm, bottom=3cm, right=1.5cm, left=1.0cm,
            headheight=2.2cm, reversemp, includemp, marginparwidth=4.5cm]{geometry}

% --- Default font ------------------------------------------------------------
% \renewcommand\familydefault{\sfdefault}

% --- Style -------------------------------------------------------------------
\renewcommand{\bibfont}{\small \sffamily}
\renewcommand{\captionfont}{\small\sffamily}
\renewcommand{\captionlabelfont}{\bfseries}

% --- Section/SubSection/SubSubSection ----------------------------------------
\titleformat{\section}
  {\normalfont\sffamily\Large\bfseries}
  {}{0pt}{}
\titleformat{\subsection}
  {\normalfont\sffamily\large\bfseries}
  {}{0pt}{}
\titleformat{\subsubsection}
  {\normalfont\sffamily\bfseries}
  {}{0pt}{}
\titleformat*{\paragraph}
  {\sffamily\normalsize}


% --- Header / Footer ---------------------------------------------------------
\usepackage{fancyhdr}
\pagestyle{fancy}
\fancyhf{}
%\renewcommand{\headrulewidth}{0.50pt}
\renewcommand{\headrulewidth}{0pt}
\fancyhead[L]{\hspace{-0.75cm}\includegraphics[width=5.5cm]{C:/Users/rfisher/Rpackages/rticles/rmarkdown/templates/joss/resources/JOSS-logo.png}}
\fancyhead[C]{}
\fancyhead[R]{}
\renewcommand{\footrulewidth}{0.25pt}

\fancyfoot[L]{\footnotesize{\sffamily Fisher \& Barneche, (). bayesnec:
An R package for Concentration-Response modelling and estimation of
No-Effect-Concentrations. \textit{Journal of Open Source Software}, (), . \href{https://doi.org/}{https://doi.org/}}}


\fancyfoot[R]{\sffamily \thepage}
\makeatletter
\let\ps@plain\ps@fancy
\fancyheadoffset[L]{4.5cm}
\fancyfootoffset[L]{4.5cm}

% --- Macros ---------

\definecolor{linky}{rgb}{0.0, 0.5, 1.0}

\newtcolorbox{repobox}
   {colback=red, colframe=red!75!black,
     boxrule=0.5pt, arc=2pt, left=6pt, right=6pt, top=3pt, bottom=3pt}

\newcommand{\ExternalLink}{%
   \tikz[x=1.2ex, y=1.2ex, baseline=-0.05ex]{%
       \begin{scope}[x=1ex, y=1ex]
           \clip (-0.1,-0.1)
               --++ (-0, 1.2)
               --++ (0.6, 0)
               --++ (0, -0.6)
               --++ (0.6, 0)
               --++ (0, -1);
           \path[draw,
               line width = 0.5,
               rounded corners=0.5]
               (0,0) rectangle (1,1);
       \end{scope}
       \path[draw, line width = 0.5] (0.5, 0.5)
           -- (1, 1);
       \path[draw, line width = 0.5] (0.6, 1)
           -- (1, 1) -- (1, 0.6);
       }
   }

% --- Title / Authors ---------------------------------------------------------
% patch \maketitle so that it doesn't center
\patchcmd{\@maketitle}{center}{flushleft}{}{}
\patchcmd{\@maketitle}{center}{flushleft}{}{}
% patch \maketitle so that the font size for the title is normal
\patchcmd{\@maketitle}{\LARGE}{\LARGE\sffamily}{}{}
% patch the patch by authblk so that the author block is flush left
\def\maketitle{{%
  \renewenvironment{tabular}[2][]
    {\begin{flushleft}}
    {\end{flushleft}}
  \AB@maketitle}}
\makeatletter
\renewcommand\AB@affilsepx{ \protect\Affilfont}
%\renewcommand\AB@affilnote[1]{{\bfseries #1}\hspace{2pt}}
\renewcommand\AB@affilnote[1]{{\bfseries #1}\hspace{3pt}}
\makeatother
\renewcommand\Authfont{\sffamily\bfseries}
\renewcommand\Affilfont{\sffamily\small\mdseries}
\setlength{\affilsep}{1em}


\ifnum 0\ifxetex 1\fi\ifluatex 1\fi=0 % if pdftex
  \usepackage[T1]{fontenc}
  \usepackage[utf8]{inputenc}

\else % if luatex or xelatex
  \ifxetex
    \usepackage{mathspec}
  \else
    \usepackage{fontspec}
  \fi
  \defaultfontfeatures{Ligatures=TeX,Scale=MatchLowercase}

\fi
% use upquote if available, for straight quotes in verbatim environments
\IfFileExists{upquote.sty}{\usepackage{upquote}}{}
% use microtype if available
\IfFileExists{microtype.sty}{%
\usepackage{microtype}
\UseMicrotypeSet[protrusion]{basicmath} % disable protrusion for tt fonts
}{}

\usepackage{hyperref}
\hypersetup{unicode=true,
            pdftitle={bayesnec: An R package for Concentration-Response modelling and estimation of No-Effect-Concentrations},
            pdfborder={0 0 0},
            breaklinks=true}
\urlstyle{same}  % don't use monospace font for urls
\usepackage{color}
\usepackage{fancyvrb}
\newcommand{\VerbBar}{|}
\newcommand{\VERB}{\Verb[commandchars=\\\{\}]}
\DefineVerbatimEnvironment{Highlighting}{Verbatim}{commandchars=\\\{\}}
% Add ',fontsize=\small' for more characters per line
\usepackage{framed}
\definecolor{shadecolor}{RGB}{248,248,248}
\newenvironment{Shaded}{\begin{snugshade}}{\end{snugshade}}
\newcommand{\AlertTok}[1]{\textcolor[rgb]{0.94,0.16,0.16}{#1}}
\newcommand{\AnnotationTok}[1]{\textcolor[rgb]{0.56,0.35,0.01}{\textbf{\textit{#1}}}}
\newcommand{\AttributeTok}[1]{\textcolor[rgb]{0.77,0.63,0.00}{#1}}
\newcommand{\BaseNTok}[1]{\textcolor[rgb]{0.00,0.00,0.81}{#1}}
\newcommand{\BuiltInTok}[1]{#1}
\newcommand{\CharTok}[1]{\textcolor[rgb]{0.31,0.60,0.02}{#1}}
\newcommand{\CommentTok}[1]{\textcolor[rgb]{0.56,0.35,0.01}{\textit{#1}}}
\newcommand{\CommentVarTok}[1]{\textcolor[rgb]{0.56,0.35,0.01}{\textbf{\textit{#1}}}}
\newcommand{\ConstantTok}[1]{\textcolor[rgb]{0.00,0.00,0.00}{#1}}
\newcommand{\ControlFlowTok}[1]{\textcolor[rgb]{0.13,0.29,0.53}{\textbf{#1}}}
\newcommand{\DataTypeTok}[1]{\textcolor[rgb]{0.13,0.29,0.53}{#1}}
\newcommand{\DecValTok}[1]{\textcolor[rgb]{0.00,0.00,0.81}{#1}}
\newcommand{\DocumentationTok}[1]{\textcolor[rgb]{0.56,0.35,0.01}{\textbf{\textit{#1}}}}
\newcommand{\ErrorTok}[1]{\textcolor[rgb]{0.64,0.00,0.00}{\textbf{#1}}}
\newcommand{\ExtensionTok}[1]{#1}
\newcommand{\FloatTok}[1]{\textcolor[rgb]{0.00,0.00,0.81}{#1}}
\newcommand{\FunctionTok}[1]{\textcolor[rgb]{0.00,0.00,0.00}{#1}}
\newcommand{\ImportTok}[1]{#1}
\newcommand{\InformationTok}[1]{\textcolor[rgb]{0.56,0.35,0.01}{\textbf{\textit{#1}}}}
\newcommand{\KeywordTok}[1]{\textcolor[rgb]{0.13,0.29,0.53}{\textbf{#1}}}
\newcommand{\NormalTok}[1]{#1}
\newcommand{\OperatorTok}[1]{\textcolor[rgb]{0.81,0.36,0.00}{\textbf{#1}}}
\newcommand{\OtherTok}[1]{\textcolor[rgb]{0.56,0.35,0.01}{#1}}
\newcommand{\PreprocessorTok}[1]{\textcolor[rgb]{0.56,0.35,0.01}{\textit{#1}}}
\newcommand{\RegionMarkerTok}[1]{#1}
\newcommand{\SpecialCharTok}[1]{\textcolor[rgb]{0.00,0.00,0.00}{#1}}
\newcommand{\SpecialStringTok}[1]{\textcolor[rgb]{0.31,0.60,0.02}{#1}}
\newcommand{\StringTok}[1]{\textcolor[rgb]{0.31,0.60,0.02}{#1}}
\newcommand{\VariableTok}[1]{\textcolor[rgb]{0.00,0.00,0.00}{#1}}
\newcommand{\VerbatimStringTok}[1]{\textcolor[rgb]{0.31,0.60,0.02}{#1}}
\newcommand{\WarningTok}[1]{\textcolor[rgb]{0.56,0.35,0.01}{\textbf{\textit{#1}}}}
\usepackage{graphicx,grffile}
\makeatletter
\def\maxwidth{\ifdim\Gin@nat@width>\linewidth\linewidth\else\Gin@nat@width\fi}
\def\maxheight{\ifdim\Gin@nat@height>\textheight\textheight\else\Gin@nat@height\fi}
\makeatother
% Scale images if necessary, so that they will not overflow the page
% margins by default, and it is still possible to overwrite the defaults
% using explicit options in \includegraphics[width, height, ...]{}
\setkeys{Gin}{width=\maxwidth,height=\maxheight,keepaspectratio}
\IfFileExists{parskip.sty}{%
\usepackage{parskip}
}{% else
\setlength{\parindent}{0pt}
\setlength{\parskip}{6pt plus 2pt minus 1pt}
}
\setlength{\emergencystretch}{3em}  % prevent overfull lines
\providecommand{\tightlist}{%
  \setlength{\itemsep}{0pt}\setlength{\parskip}{0pt}}
\setcounter{secnumdepth}{0}
% Redefines (sub)paragraphs to behave more like sections
\ifx\paragraph\undefined\else
\let\oldparagraph\paragraph
\renewcommand{\paragraph}[1]{\oldparagraph{#1}\mbox{}}
\fi
\ifx\subparagraph\undefined\else
\let\oldsubparagraph\subparagraph
\renewcommand{\subparagraph}[1]{\oldsubparagraph{#1}\mbox{}}
\fi

% Pandoc citation processing
\newlength{\csllabelwidth}
\setlength{\csllabelwidth}{3em}
\newlength{\cslhangindent}
\setlength{\cslhangindent}{1.5em}
% for Pandoc 2.8 to 2.10.1
\newenvironment{cslreferences}%
  {\setlength{\parindent}{0pt}%
  \everypar{\setlength{\hangindent}{\cslhangindent}}\ignorespaces}%
  {\par}
% For Pandoc 2.11+
\newenvironment{CSLReferences}[3] % #1 hanging-ident, #2 entry spacing
 {% don't indent paragraphs
  \setlength{\parindent}{0pt}
  % turn on hanging indent if param 1 is 1
  \ifodd #1 \everypar{\setlength{\hangindent}{\cslhangindent}}\ignorespaces\fi
  % set entry spacing
  \ifnum #2 > 0
  \setlength{\parskip}{#2\baselineskip}
  \fi
 }%
 {}
\usepackage{calc} % for calculating minipage widths
\newcommand{\CSLBlock}[1]{#1\hfill\break}
\newcommand{\CSLLeftMargin}[1]{\parbox[t]{\csllabelwidth}{#1}}
\newcommand{\CSLRightInline}[1]{\parbox[t]{\linewidth - \csllabelwidth}{#1}}
\newcommand{\CSLIndent}[1]{\hspace{\cslhangindent}#1}


\title{bayesnec: An R package for Concentration-Response modelling and
estimation of No-Effect-Concentrations}

        \author[1, 2]{Rebecca Fisher}
          \author[1, 2]{Diego Barneche}
    
      \affil[1]{Australian Institute of Marine Science, Australia}
      \affil[2]{}
  \date{\vspace{-5ex}}

\begin{document}
\maketitle

\marginpar{
  %\hrule
  \sffamily\small

  {\bfseries DOI:} \href{https://doi.org/}{\color{linky}{}}

  \vspace{2mm}

  {\bfseries Software}
  \begin{itemize}
    \setlength\itemsep{0em}
    \item \href{}{\color{linky}{Review}} \ExternalLink
    \item \href{}{\color{linky}{Repository}} \ExternalLink
    \item \href{}{\color{linky}{Archive}} \ExternalLink
  \end{itemize}

  \vspace{2mm}

  {\bfseries Submitted:} \\
  {\bfseries Published:} 

  \vspace{2mm}
  {\bfseries License}\\
  Authors of papers retain copyright and release the work under a Creative Commons Attribution 4.0 International License (\href{http://creativecommons.org/licenses/by/4.0/}{\color{linky}{CC-BY}}).
}

\hypertarget{summary}{%
\section{Summary}\label{summary}}

The \texttt{bayesnec} is an R package to fit concentration(dose) -
response curves to toxicity data, and derive No-Effect-Concentration
(\emph{NEC}), No-Significant-Effect-Concentration (\emph{NSEC}), and
Effect-Concentration (of specified percentage `x', \emph{ECx})
thresholds from non-linear models fitted using Bayesian MCMC fitting
methods via \texttt{brms} (Bürkner 2017; Bürkner 2018) and
\texttt{stan}. The package is an adaptation and extension of an initial
package \texttt{jagsNEC} (Fisher, Ricardo, and Fox 2020) which was based
on the \texttt{R2jags} package (Su and Yajima 2015) and \texttt{jags}
(Plummer 2003). In \texttt{bayesnec} it is possible to fit a single
model, custom model set, specific model set or all of the available
models. When multiple models are specified the \texttt{bnec} function
returns a model weighted estimate of predicted posterior values. A range
of support functions and methods are also included to work with the
returned single, or multi- model objects that allow extraction of raw or
model averaged predicted, ECx, NEC and NSEC values and to interrogate
the fitted model or model set.

\hypertarget{statement-of-need}{%
\section{Statement of need}\label{statement-of-need}}

Concentration-response (C-R) modelling is fundamental to assessing
toxicity and deriving toxicity thresholds used in the risk assessments
that underpin protection of human health and the environment. They are
widespread in the disciplines of pharmacology, toxicology and
ecotoxicoloy. Typically C-R curves are non-linear, which increases the
complexity of their numerical estimation. Additionally, estimates of
uncertainty in parameters and derived thresholds are critical to
effective integration in risk assessment and formal decision frameworks.
Bayesian methods that allow robust quantification of uncertainty with
intuitive and direct probabilistic meaning are therefore ideal tools C-R
modelling in most settings.

Bayesian model fitting can be difficult to automate across a broad range
of usage cases, particularly with respect to specifying valid initial
values and appropriate priors. This is one reason the use of Bayesian
statistics for \emph{NEC} estimation (or even \emph{ECx} estimation) is
not currently widely adopted across the broader ecotoxicological and
toxicology community, who rarely have access to specialist statistical
expertise. The \texttt{bayesnec} package provides an accessible
interface specifically for fitting \emph{NEC} models and other
concentration-response models using Bayesian methods. A range of models
can be specified based on the known distribution of the
``concentration'' or ``dose'' variable (the predictor, x) as well as the
``response'' (y) variable. The model formula, including priors and
initial values required to call a \texttt{brms} model are automatically
generated based on information contained in the supplied data.

\hypertarget{development-and-functionality}{%
\section{Development and
Functionality}\label{development-and-functionality}}

This project started with an implementation of the \emph{NEC} model
based on that described in (Fox 2010) using R2jags. The package has been
further generalised to allow a large range of response variables to be
modelled using the appropriate statistical distribution. While the
original \texttt{jagsNEC} implementation supported gaussian, poisson,
binomial, gamma, negbin and beta response data \texttt{bayesnec}
supports any of the \texttt{brms} families. We have since also further
added a range of alternative \emph{NEC} model types, as well as a range
of typically used concentration-response models (such as 4-parameter
logistic and weibull models) that have no \emph{NEC} `step' function but
simply model the response as a smooth function of concentration, as can
be fit using other commonly used frequentist packages such as
\texttt{drc} (Ritz et al. 2016).

The main working function in \texttt{bayesnec} is \texttt{bnec} Specific
models can be fit directly using \texttt{bnec}, in which case an object
of class \texttt{bayesnecfit} is returned. Alternatively it is possible
to fit a custom model set, specific model set or all of the available
models. Were multiple models are fit a \texttt{bayesmanecfit} model
object is returned, and a model averaging approached is used to
synthesise the returned output (see below for more details).

We have attempted to make the \texttt{bnec} function as easy to use as
possible, targeting the novice R user, whilst ensuring the important
relevant arguments can be set and modified as required by those more
advanced. Only three arguments must be supplied, including:
\texttt{data} - A \texttt{data.frame} containing the data to use for the
model; \texttt{x\_var} - A \texttt{character} indicating the column
heading containing the concentration (x) variable; and \texttt{y\_var} -
A \texttt{character} indicating the column heading containing the
response (y) variable. While the distribution of the x and y variables
can be specified directly, \texttt{bayesnec} will automatically `guess'
the correct distribution to use based on the characteristics of the
provided data, as well as build the relevant model priors and initial
values for the \texttt{brms} model.

The statistical families available for modelling the response variabe
\texttt{y\_var} include binomial, beta, betabinomial, poisson, gamma,
gaussian, and negative binomal. If not supplied, the appropriate
distribution will be guessed based on the characteristics of the input
data through \texttt{check\_data}. Guesses include all of the available
families except negbinomial and betabinomimal because these requires
knowledge on whether the data is over-dispersed.

\hypertarget{model-diagnostics}{%
\subsection{Model diagnostics}\label{model-diagnostics}}

\texttt{bayesnec} includes a summary method for both
\texttt{bayesnecfit} and \texttt{bayesmanecfit} model objects the
provides the usual summary of the model a parameters, along with any
relevant model fits statistics as returned in the underlying
\texttt{brm} model fits.

A range of tools are available to the user to assess model fit,
including an estimate of overdispersion (for relevant families), an
extension of the \texttt{brms} \texttt{rhat} function that can be
applied both \texttt{bayesnecfit} and \texttt{bayesmanecfit} model
objects and a function \texttt{check\_chains} that can be used to
visually assess chain mixing.

Several helper functions have been included that allow the user to add
or drop models from a \texttt{bayesmanecfit} object , or change the
model weighting method (\texttt{amend}); extract a single or subset of
models from the \texttt{bayesmanecfit} object (\texttt{pull\_out}); and
examine the priors use for model fitting (\texttt{pull\_prior} and
\texttt{sample\_prior}).

\hypertarget{model-inference}{%
\subsection{Model inference}\label{model-inference}}

Base R (\texttt{plot}) and ggplot2 (\texttt{ggbnec}) plotting methods,
as well as predict methods have also been developed for both
\texttt{bayesnecfit} and \texttt{bayesmanecfit} model classes. In
addition, there are method based functions for extracting ECx
(\texttt{ecx}), NEC (\texttt{nec}) and NSEC (\texttt{nsec}) threshold
values. In all cases the posterior samples that underpin these functions
are achieved through \texttt{posterior\_epred} from the \texttt{brms}
package.

\hypertarget{models-in-bayesnec}{%
\section{\texorpdfstring{Models in
\texttt{bayesnec}}{Models in bayesnec}}\label{models-in-bayesnec}}

There are a range of models available in \texttt{bayesnec}.

The argument \texttt{model} in the function \texttt{bnec} is a character
string indicating the name(s) of the desired model (see ?models for more
details, and the list of models available). If a recognised model name
is provided a single model of the specified type is fit, and
\texttt{bnec} returns a model object of class \texttt{bayesnecfit}. If a
vector of two or more of the available models are supplied,
\texttt{bnec} returns a model object of class \texttt{bayesmanecfit}
containing Bayesian model averaged predictions for the supplied models,
providing they were successfully fitted.

The \texttt{model} argument may also be one of ``all'', meaning all of
the available models will be fit; ``ecx'' meaning only models excluding
the \(\eta = \text{NEC}\) step parameter will be fit; or ``nec'' meaning
only models with the \(\eta = \text{NEC}\) step parameter (see below
\textbf{Model parameters}) will be fit. There are a range of other
pre-define model groups available. The full list of currently
implemented model groups can be seen using:

\begin{Shaded}
\begin{Highlighting}[]
\KeywordTok{library}\NormalTok{(}\StringTok{"bayesnec"}\NormalTok{)}
\KeywordTok{models}\NormalTok{()}
\end{Highlighting}
\end{Shaded}

\hypertarget{parameter-definitions}{%
\subsection{Parameter definitions}\label{parameter-definitions}}

Where possible we have aimed for consistency in the interpretable
meaning of the individual parameters across models. Across the currently
implemented model set, models contain from two (basic linear or
exponential decay, see \textbf{ecxlin} or \textbf{ecxexp}) to five
possible parameters (\textbf{nechorme4}), including:
\(\tau = \text{top}\), usually interpretable as either the y-intercept
or the upper plateau representing the mean concentration of the response
at zero concentration; \(\eta = \text{NEC}\), the
no-effect-concentration value (the x concentration value where the
breakpoint in the regression is estimated at, see \textbf{Model types
for NEC and ECx estimation} and (Fox 2010) for more details on parameter
based NEC estimation); \(\beta = \text{beta}\), generally the
exponential decay rate of response, either from 0 concentration or from
the estimated \(\eta\) value, with the exception of the
\textbf{neclinhorme} model where it represents a linear decay from
\(\eta\) because slope (\(\epsilon\)) is required for the linear
increase; \(\delta = \text{bottom}\), representing the lower plateau for
the response at infinite concentration; \(\alpha = \text{slope}\), the
linear decay rate in the models \textbf{neclin} and \textbf{ecxlin}, or
the linear increase rate prior to \(\eta\) for all hormesis models; and
\(\epsilon = \text{d}\), the exponent in the \textbf{ecxsigm} and
\textbf{necisgm} models.

In addition to the model parameters, all \texttt{nec...} models have a
step function used to define the breakpoint in the regression, which can
be defined as

\[
f(x_i, \eta) = \begin{cases} 
      x_i - \eta < 0, & 0 \\
      x_i - \eta \geq 0, & 1 \\
   \end{cases}
\]

\hypertarget{model-types-for-nec-and-ecx-estimation}{%
\subsection{Model types for NEC and ECx
estimation}\label{model-types-for-nec-and-ecx-estimation}}

All models provide an estimate for the No-Effect-Concentration (NEC).
For model types with ``nec'' as a prefix, the NEC is directly estimated
as parameter \(\eta = \text{NEC}\) in the model, as per (Fox 2010).
Models with ``ecx'' as a prefix are continuous curve models, typically
used for extracting ECx values from concentration response data. In this
instance the NEC reported is actually the
No-Significant-Effect-Concentration (NSEC), defined as the concentration
at which there is a user supplied (see \texttt{sig\_val}) percentage
certainty (based on the Bayesian posterior estimate) that the response
falls below the estimated value of the upper asymptote
(\(\tau = \text{top}\)) of the response (i.e.~the response value is
significantly lower than that expected in the case of no exposure). The
default value for \texttt{sig\_val} is 0.01, which corresponds to an
alpha value (Type 1 error rate) of 0.01 for a one-sided test of
significance. See \texttt{?nsec} for more details. We currently
recommend only using the ``nec'' model set for estimation of NEC values,
as the NSEC concept has yet to be formally peer-reviewed.

ECx estimates can be equally validly obtained from both ``nec'' and
``ecx'' models. ECx estimates will usually be lower (more conservative)
for ``ecx'' models fitted to the same data as ``nec'' models (see the
{[}Comparing posterior predictions{]}{[}e4{]} vignette for an example.
However, we recommend using ``all'' models where ECx estimation is
required because ``nec'' models can fit some datasets better than
``ecx'' models and the model averaging approach will place the greatest
weight for the outcome that best fits the supplied data. This approach
will yield ECx estimates that are the most representative of the
underlying relationship in the dataset.

There is ambiguity in the definition of ECx estimates from hormesis
models (these allow an initial increase in the response (see (Mattson
2008)) and include models with the character string \texttt{horme} in
their name), as well as those that have no natural lower bound on the
scale of the response (models with the string \texttt{lin} in their
name, in the case of gaussian response data). For this reason the
\texttt{ecx} function has arguments \texttt{hormesis\_def} and
\texttt{type}, both character vectors indicating the desired behaviour.
For \texttt{hormesis\_def\ =\ "max"} ECx values are calculated as a
decline from the maximum estimates (i.e.~the peak at
\(\eta = \text{NEC}\)); and \texttt{hormesis\_def\ =\ "control"} (the
default) indicates ECx values should be calculated relative to the
control, which is assumed to be the lowest observed concentration. For
\texttt{type\ =\ "relative"} ECx is calculated as the percentage
decrease from the maximum predicted value of the response
(\(\tau = \text{top}\)) to the minimum predicted value of the response
(ie, `relative' to the observed data). For \texttt{type\ =\ "absolute"}
(the default) ECx is calculated as the percentage decrease from the
maximum value of the response (\(\tau = \text{top}\)) to 0 (or
\(\delta = \text{bottom}\) for models with that parameter). For
\texttt{type\ =\ "direct"} a direct interpolation of y on x is obtained.

\hypertarget{model-suitability-for-response-types}{%
\subsection{Model suitability for response
types}\label{model-suitability-for-response-types}}

Models that have an exponential decay (most models with parameter
\(\beta = \text{beta}\)) with no \(\delta = \text{bottom}\) parameter
are \texttt{0} bounded and are not suitable for the gaussian family, or
any family modelled using a logit or log link because they cannot
generate predictions of negative y (response). Conversely models with a
linear decay (containing the string ``lin'' in their name) are not
suitable for modelling families that are \texttt{0} bounded (gamma,
poisson, negative binomial) using an identity link. Models with a linear
decay or hormesis linear increase (all models with parameter
\(\alpha = \text{slope}\)) are not suitable for modelling families that
are \texttt{0}, \texttt{1} bounded (binomial, beta and betabinomial2).
These restrictions do not need to be controlled by the user as a call to
\texttt{bnec} with \texttt{models\ =\ "all"} will simply exclude
inappropriate models, albeit with a warning.

\hypertarget{model-averaging-and-multimodel-inference}{%
\section{Model averaging and multimodel
inference}\label{model-averaging-and-multimodel-inference}}

When multiple models are fit using \texttt{bnec} an object of class
\texttt{bayesmanecfit} is returned that includes a model weighted
estimate of predicted posterior values. \texttt{bayesnec} also includes
a set of \texttt{bayesmanecfit} methods for predicting, plotting and
extracting effect concentrations and related threshold values that
return model weighted estimates.

Multi-model inference can be useful where there are a range of plausible
models that could be used (Burnham and Anderson 2002) and has been
recently adopted in ecotoxicology for SSD model inference (Thorley and
Schwarz 2018). The approach may have considerable value in
concentration-response modelling because there is often no a priori
knowledge of the functional form that the response relationship should
take. In this case model averaging can be a useful way of allowing the
data to drive the model selection processing, with weights proportional
to how well the individual models fits the data. Well fitting models
will have high weights, dominating the model averaged outcome.
Conversely, poorly fitting models will have very low model weights and
will therefore have little influence on the outcome. Were multiple
models fit the data equally well, these can equally influence the
outcome, and the resultant posterior predictions reflect that model
uncertainty. It is possible to specify the ``stacking'' method (Yao et
al. 2018) for model weights if desired (through the argument
\texttt{loo\_controls}) which aim to minimise prediction error. We do
not currently recommend using stacking weights given the typical sample
sizes associated with most concentration-response experiments, and
because the main motivation for model averaging within the
\texttt{bayesnec} package is to properly capture model uncertainty
rather than reduce prediction error. By default
\texttt{model\ weights}bnec\texttt{uses\ the\ \ "pseudobma"\ method\ with\ Bayesian\ bootstrap\ through}loo\_model\_weights`
(Vehtari et al. 2020; Vehtari, Gelman, and Gabry 2017). These are
reasonably analogous to the way model weights are generated using AIC or
AICc (Burnham and Anderson 2002).

\hypertarget{model-comparison}{%
\section{Model comparison}\label{model-comparison}}

With \texttt{bayesnec} we have included a function
(\texttt{compare\_posterior}) that allows bootstrapped comparisons of
posterior predictions. This function allows the user to fit several
different \texttt{bnec} model fits and can compare differences in the
posterior predictions across these fits for individual endpoint
estimates (e.g.~nec, nsec or ecx) or across a range of predictor (x)
values. While usage is demonstrated in the relevant vignette by
comparing different types of models and model sets using a single
dataset. However, the intent of this function is to allow comparison
across different datasets that might represent, for example, different
levels of a fixed factor covariate. At this time \texttt{bnec} does not
allow inclusion of an interaction with a fixed factor. Including an
interaction term within each of the non-linear models implemented in
\texttt{bayesnec} is relatively straightforward, and may be introduced
in future releases. However, in many cases the functional form of the
response may change with different levels of a given factor. The
substantial complexity of defining all possible non-linear model
combinations at each factor level means it unlikely this could be
feasibly implemented in \texttt{bayesnec} in the short term. In the
meantime the greatest flexibility in the functional form of individual
model fits can be readily obtained using models fitted independently to
data within each factor level.

\hypertarget{acknowledgements}{%
\section{Acknowledgements}\label{acknowledgements}}

The development of \texttt{bayesnec} was supported by an AIMS internal
grant. David Fox and Gerard Ricardo developed some of the initial code
on which the \texttt{bayesnec} predecessor \texttt{jagsNEC} was based.
Usage, testing and functionality of both the \texttt{jagsNEC} and
\texttt{bayesnec} packages was substantially aided through input from
Joost van Dam, Andrew Negri, Florita Florence, Heidi Luter, Marie Thomas
and Mikaela Nordborg.

\hypertarget{references}{%
\section*{References}\label{references}}
\addcontentsline{toc}{section}{References}

\hypertarget{refs}{}
\begin{cslreferences}
\leavevmode\hypertarget{ref-Burnham2002}{}%
Burnham, K P, and D R Anderson. 2002. \emph{Model Selection and
Multimodel Inference; A Practical Information-Theoretic Approach}. 2nd
ed. New York: Springer.

\leavevmode\hypertarget{ref-Burkner2017}{}%
Bürkner, Paul Christian. 2017. ``brms: An R package for Bayesian
multilevel models using Stan.'' \emph{Journal of Statistical Software}.
\url{https://doi.org/10.18637/jss.v080.i01}.

\leavevmode\hypertarget{ref-Burkner2018}{}%
Bürkner, Paul-Christian. 2018. ``Advanced Bayesian Multilevel Modeling
with the R Package brms.'' \emph{The R Journal} 10 (1): 395--411.
\url{https://doi.org/10.32614/RJ-2018-017}.

\leavevmode\hypertarget{ref-Fisher2020}{}%
Fisher, Rebecca, Gerard Ricardo, and David Fox. 2020. ``Bayesian
concentration-response modelling using jagsNEC.''
\url{https://doi.org/10.5281/ZENODO.3966864}.

\leavevmode\hypertarget{ref-Fox2010}{}%
Fox, David R. 2010. ``A Bayesian approach for determining the no effect
concentration and hazardous concentration in ecotoxicology.''
\emph{Ecotoxicology and Environmental Safety} 73 (2): 123--31.

\leavevmode\hypertarget{ref-Mattson2008}{}%
Mattson, Mark P. 2008. ``Hormesis defined.''
\url{https://doi.org/10.1016/j.arr.2007.08.007}.

\leavevmode\hypertarget{ref-Plummer2003}{}%
Plummer, Martyn. 2003. `` JAGS: A program for analysis of Bayesian
graphical models using Gibbs sampling.
http://citeseer.ist.psu.edu/plummer03jags.html.''

\leavevmode\hypertarget{ref-Ritz2016}{}%
Ritz, Christian, Florent Baty, Jens C Streibig, and Daniel Gerhard.
2016. ``Dose-Response Analysis Using R.'' \emph{PLoS ONE} 10 (12):
e0146021. \url{https://doi.org/10.1371/journal.pone.0146021}.

\leavevmode\hypertarget{ref-Su2015}{}%
Su, Yu-Sung, and Masanao Yajima. 2015. ``R2jags: Using R to Run 'JAGS'.
R package version 0.5-6. http://CRAN.R-project.org/package=R2jags.''

\leavevmode\hypertarget{ref-Thorley2018}{}%
Thorley, Joe, and Carl Schwarz. 2018. ``ssdtools: Species Sensitivity
Distributions. R package version 0.0.3.
https://CRAN.R-project.org/package=ssdtools.''

\leavevmode\hypertarget{ref-vehtari2020}{}%
Vehtari, Aki, Jonah Gabry, Mans Magnusson, Yuling Yao, Paul-Christian
Bürkner, Topi Paananen, and Andrew Gelman. 2020. ``Loo: Efficient
Leave-One-Out Cross-Validation and Waic for Bayesian Models.''
\url{https://mc-stan.org/loo}.

\leavevmode\hypertarget{ref-vehtari2017}{}%
Vehtari, Aki, Andrew Gelman, and Jonah Gabry. 2017. ``Practical Bayesian
Model Evaluation Using Leave-One-Out Cross-Validation and Waic.''
\emph{Statistics and Computing} 27 (5): 1413--32.
\url{https://doi.org/10.1007/s11222-016-9696-4}.

\leavevmode\hypertarget{ref-Yao2018}{}%
Yao, Yuling, Aki Vehtari, Daniel Simpson, and Andrew Gelman. 2018.
``Using Stacking to Average Bayesian Predictive Distributions (with
Discussion).'' \emph{Bayesian Analysis}.
\url{https://doi.org/10.1214/17-ba1091}.
\end{cslreferences}

\end{document}
